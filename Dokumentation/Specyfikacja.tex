\documentclass[a4paper, 12pt]{article}

\usepackage[T1]{fontenc}
\usepackage[polish]{babel} 
\usepackage[utf8]{inputenc} 
\let\lll\undefined
\usepackage{setspace}
\usepackage{fancyhdr}
\usepackage{hyperref}
\usepackage{pdfpages}
\usepackage{listings}
\usepackage{color}
\usepackage{graphicx}
\usepackage{enumitem}
\usepackage{latexsym}
\pagestyle{fancy} 
\hypersetup{
    colorlinks=true,
    linkcolor=blue,
    filecolor=magenta,      
    urlcolor=cyan,
}
\newcommand{\mainmatter}{\clearpage \cfoot{\thepage\ of \pageref{LastPage}}
\pagenumbering{arabic}} 
\lstset{language=bash}  
\begin{document}

	\begin{titlepage}
\includegraphics[width = 40mm]{logo.jpg}
		\begin{center}
    			\vspace{3cm}
    					\Large\textit{\textbf{Rozproszony algorytm genetyczny  do wyszukiwania globalnej ekstremum: MPI}}
    					\Large\textit{\textbf{Programowanie równoległe i rozproszone}}
   			\vspace{4cm}
		\end{center} 

		\hfill\begin{minipage}{0.54\textwidth}
			\Large Wykonanie:\newline
				1. Ivan Prakapets  \newline
				2. Piotr Jeleniewicz 
		\vspace{\baselineskip}
		\end{minipage}
		
		\hfill\begin{minipage}{0.54\textwidth}
			\Large Sprawdzająca:\newline
		 		dr inż. Zuzanna Krawczyk
\vspace{\baselineskip}
		\end{minipage}
	
		\hfill\begin{minipage}{0.7\textwidth}
		\vspace{1cm}
			\Large Warszawa, 2020
			\vspace{\baselineskip}
		\end{minipage}
	\end{titlepage}
\newpage
\mainmatter
\setlength{\headheight}{15pt}
\doublespacing
\tableofcontents
\newpage

\linespread{0.5}
\setlist{nolistsep}

\section{Opis problemu}
\hspace*{1cm} Celem projektu jest stworzenie  i implementacja  rozproszonego algorytmu genetycznego, którego celem jest wyszukiwanie globalnych ekstremów. 
\section{Opis funkcjonalności programu}
Aplikacja będzie udostępniać następujące funkcjonalności:

\begin{itemize}
    \item Wprowadzanie badanej funkcji funkcji
    \item Wprowadzanie liczby wejściowej 
    \item Wprowadzanie liczby generacji
    \item Wprowadzanie optymalizacji wejścia ( min lub max )
    \item Wizualizacja algorytmu
    \item Wyszukiwanie minimum i maksimum funkcji z dwóch zmiennych
    \item Pokazanie mutacji
    \item Zapisanie wyników do do pliku
    \item Wyświetlenie  statystyk
    \item Wyświetlenie  grafiki z wynikami
\end{itemize}

\section{Analiza możliwości zrównoleglenia programu}
\hspace*{1cm} W algorytm genetycznym można zrównoleglić tworzenie nowych pokoleń na podstawie poprzednio wygenerowanych.
  
\section{Wybór technologii/języka/biblioteki}
	\hspace*{1cm} Projekt zostanie napisany przy użyciu języka Python oraz biblioteki MPI. 
Oprócz tego wykorzystane zostaną następujące biblioteki:
\begin{itemize}
    \item MPI do zrównoleglenia programu
    \item numpy, pandas
    \item tkinter dla wizualizacji
    \item matpltolib do wyświetlenie grafików
\end{itemize}
\section{Opis sposobu zrównoleglenia}
Program będzie działał w sposób rozproszony - instancja główna aplikacji będzie rozsyłać do aplikacji wykonwaczych dane, a następnie odbierać je i łączyć w finalny wynik. Każda z aplikacji wykonawczych będzie mogła procesować wykorzystując kilka procesów. 
\section{Wnioski} 
\begin{enumerate}
    \item Właściwy zestaw parametrów pozwala na dość dokładne znalezienie globalnego ekstremum funkcji z dwóch zmiennych
    \item Wizualizacja może pomóc w debugowaniu i ulepszaniu algorytmu, a także pomaga zrozumieć przebieg i zasady samego algorytmu genetycznego
\end{enumerate}

\label{LastPage}~
\label{LastPageOfBackMatter}~		
\end{document}